\section{一般条款}

\subsection{不可抗力}
\begin{enumerate}
  \item 本合同生效后,一方因地震、台风、水灾、战争、疫情及其他不能预见、不能避免并不能克服的不可抗力事件导致不能履行本合同的,根据不可抗力的影响,部分或者全部免除违约责任。但本款并不免除甲方支付乙方已提供服务所应得到的技术服务费用的责任。
  \item 不可抗力发生后,遭遇不可抗力的一方应尽快通知对方,双方应采取适当措施防止损失的扩大。一方如果没有采取适当措施致使损失扩大的,应就扩大的损失承担赔偿责任。
  \item 不可抗力的影响消除后,遭遇不可抗力的一方应在 10 日内向对方提供公证机关等有权机关出具的证明。
  \item 不可抗力的影响消除后,除非双方已经达成其他书面协议,否则遭遇不可抗力的一方应继续履行本合同。如果不可抗力的影响持续达30日,双方应尽快协商签订补充协议或终止协议。
\end{enumerate}

\subsection{陈述和保证}
甲、乙双方均保证其为合法成立并持续经营的中华人民共和国法人,其已获得有关政府部门的审批(如需)和/或其他必要授权签订并履行本合同,其签订并履行本合同不会违反对其具有拘束力的其他文件。

\subsection{双方关系}
\begin{enumerate}
  \item 本合同项下任何条款不应被解释为在甲、乙双方之间创设合资、合伙、代理或本合同目的以外的任何其他关系。
  \item 为免疑义,双方理解并确认,无论本合同其他条款如何约定,本合同的签署及/或项目人员履行或提供本合同项下服务均不得被理解或解释为甲方与项目人员之间形成劳动合同法律关系,亦不得被理解或解释为甲方与乙方之间存在劳务派遣服务法律关系。项目人员的工资、社会保险、住房公积金由乙方发放、缴纳。项目人员的工资、社会保险、住房公积金由乙方发放、缴纳。
  \item 本框架合同同样适用于甲方指定的关联机构,即甲方关联机构若与乙方签署工作说明书(SOW)、工作订单(WO)等,同样适用此合同。甲方关联机构的指定以及名称、数量等有任何变动,甲方应当及时书面(包括甲方指定联系人以电子邮件形式)通知乙方。本合同履行过程中,甲方及其关联机构应分别与乙方结算。
\end{enumerate}

\subsection{独立性}
本合同的任何条款如被裁决为无效或不可执行,其他条款仍然有效。

\subsection{非弃权}
\begin{enumerate}
  \item 一方未能/迟延行使其在本合同项下的任何权利,不应被解释为其弃权。
  \item 一方放弃其在本合同项下的任何权利应以书面形式明确作出,且仅限于该等书面文件载明的范围。
  \item 一方放弃其就对方违反本合同项下特定义务的追索权,不应被解释为放弃其就对方继续违反该等义务的追索权,也不应被解释为放弃其就对方违反本合同项下其他义务的追索权。
\end{enumerate}

\subsection{完整性和优先性}
\begin{enumerate}
  \item 本合同反映并构成了双方就本合同标的达成的全部协议,取代之前所有口头及书面的惯例、意向、声明、承诺及协议。
  \item 本合同项下标题仅为方便阅读所设,并不影响本合同项下任何部分的含义或解释。
\end{enumerate}

\subsection{通知与送达}
\begin{enumerate}
  \item 除本合同另有约定外,与本合同有关的通知、要求、指令、司法文书和其他通讯应以书面形式(如无相反约定,包括电子邮件、传真)做出并且应由发出通知的一方或其授权代表签署,发送至对方联系人用于接收通知的地址和传真号码,方构成一个有效的通知。
  \item 通知送达日期按下列约定确定:
    \begin{enumerate}
      \item 如果邮寄,以快递单号查询所示签收日期视为送达日。
      \item 以传真、电传、电报传送,在收到电码或成功发送确认章的情况下,则以发出后的第 2 个工作日视为送达。
      \item 如以亲自或委托递交的方式,则以收件方联系人或授权代理人签收视为送达。
      \item 以电子邮件传送,以电子邮件进入收件人指定的电子邮件系统时视为送达。
      \item 甲乙双方如通过诉讼、仲裁等方式解决相关纠纷的,上述日期亦视为送达日期。
    \end{enumerate}
  \item 如果联系信息变化,变更方应在变更后 5 日内通知对方,否则由此造成的损失由责任方承担。
\end{enumerate}
