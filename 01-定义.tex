\section{定义}
除非双方另有书面约定或上下文另有所指,本合同中下列术语含义如下:

\subsection{服务地点}
指乙方提供服务的地点。

\subsection{关联机构}
指由一方直接或间接控制;或直接或间接控制一方;或与一方受同一机构直接或间接控制的机构。“机构”指任何自然人、法人;“控制”指直接或间接地拥有影响机构管理的能力,无论是通过所有权、有投票权的股份、协议或其他方式。

\subsection{工作日}
指中华人民共和国境内商业银行的对公营业日(不包括休息日和法定休假日)。

\subsection{合同}
指本合同正文、构成本合同不可分割部分的附件以及双方约定的其他文件。

\subsection{甲方客户}
亦称“最终用户”,指最终实际享受乙方提供的服务的主体。

\subsection{技术文档}
指本合同约定的与服务相关的技术性文件,例如图纸、手册、标准、参数以及其他文字和/或图表说明。

\subsection{验收}
指甲方按本合同约定的标准和期限,由双方派员组成的验收小组,对乙方交付的服务成果的功能进行测试、验证并出具书面报告。本术语及其含义不适用于计时计件收费类项目。

\subsection{服务成果}
指任何体现服务成果的观念、意图、设想、目标等的无形效果以及图纸、文档、模具、模型、样品、计算机存储介质等有形客体。

\subsection{软件}
指由硬连线逻辑指令及置于系统存储器内的机器可读码(包括但不限于半导体装置或系统)组成的电脑程序,可提供基本逻辑、操作指令以及与用户相关的应用程序指令,包括用于说明、维护及使用程序的有关文件。

\subsection{日,天}
指自然日。

\subsection{升级}
指乙方对服务成果所作的功能性变更,包括但不限于增减、改变功能;增减最大并发用户数量;调整用户界面。本术语及其含义不适用于计时计件收费类项目。

\subsection{生效日}
指本合同签订之日。

\subsection{书面}
包括签章以及电子邮件、电子数据交换等数据电文形式。

\subsection{双方}
指甲方和乙方。

\subsection{维护}
指自甲方签署终验合格报告之次日起的一定期限内,乙方为使服务成果具备并保持本合同约定的功能而排除故障、修正缺陷,但不包括升级。本术语及其含义不适用于计时计件收费类项目。

\subsection{项目服务}
指乙方按照本合同要求,完成约定服务并向甲方交付特定服务成果的服务。

\subsection{现场}
指乙方提供服务和/或安装服务成果的地点。

\subsection{现场维护}
指乙方在现场对服务成果进行维护。本术语及其含义不适用于计时计件收费类项目。

\subsection{项目人员}
指接受乙方指派,为甲方提供服务的乙方雇员(不包括乙方的销售人员、商务人员以及其他辅助人员)。

\subsection{一方}
指甲方或乙方。

\subsection{远程维护}
指乙方通过电话、传真、电子邮件、虚拟专用网络等非现场方式对服务成果进行维护。本术语及其含义不适用于计时计件收费类项目。

\subsection{指令性额外服务}
指甲方项目负责人要求的,或项目人员申请并经双方项目负责人书面同意的额外服务。

\subsection{中国}
指中华人民共和国,为本合同目的,不包括香港特别行政区、澳门特别行政区和台湾地区。

\subsection{实质性违约}
指根本性违约,即一方当事人违反合同的结果,如使另一方当事人蒙受实质损害,以至于实际上剥夺了一方根据合同有权期待得到的东西,如拒绝交货/收货/付款等。

